\documentclass[fleqn]{article}
\usepackage{enumitem}
\usepackage{fancyhdr}

\pagestyle{fancy}
\fancyhf{}
\lhead{Henrik Samuelsson}
\rhead{henrik.samuelsson@gmail.com}

\begin{document}

\section*{Problems Solutions Chapter 1}
Solutions for the problems in chapter 1 of the book Electronic Devices 9th edition.

\begin{enumerate}[label=\textbf{\arabic*.}]

%problem 1 
\item A neutral atom with atomic number 6 will have 6 electrons and 6 protons. 
  
%problem 2  
\item The maximum number of electrons that can exist in an given atom shell $ n $ is calculated by the formula

\[
  N = 2n^2
\]

This means that the maximum number of electronics in the third shell of an atom is
\[ 
  2 \cdot 3^2 = 2 \cdot 9 = 18
\]

%problem 3
\item
Materials are categorized into three groups called isolators, semiconductors, and conductors. Isolators have the largest band gap between the conduction band and valence band. Semiconductors will have a smaller band gap than isolators. Conductors have an overlap between the conduction band and the valence band.

%problem 4
\item
There are several types of atoms that have four valence electrons, for example silicon and germanium. These types of atoms are classified as semiconductors.

%problem 5
\item
A single atom in a silicon crystal forms four covalent bounds.

%problem 6
\item
Adding heat to silicon will cause valence electrons to become free electrons, this increases the conductivity.

%problem 7
\item
The valencen band and the conduction band are the two energy bands at which current is produced in silicon.

%problem 8
\item
Doping intentionally introduces impurities into an extremely pure intrinsic semiconductor for the purpose of modulating its electrical properties. N-doping increases the number of free electrons by adding impurity atoms that have five valence electrons. P-doping increases the number of holes by adding impurity atoms that have three valence electrons.

%problem 9
\item
Antimony is a chemical element that can be used as a dopant to create n-type silicon.

Boron is a chemical element that can be used as a dopant to create p-type silicon.

%problem 10
\item



\end{enumerate}

\end{document}